\section{John Fitzgerald Kennedy}
In 1960 John Fitzgerald Kennedy was elected president of the United States.

At forty-three he was the youngest man ever elected President of the USA, and the first one who was catholic.

He said to the Americans: <<Ask not what your country can do for you; Ask what you can do for your country>>. He promised to work for better conditions for the poor: better pay, houses and medical care. He wanted to improve education. He was ready to help people who couldn't help themselves.

Above all, he demanded equal civil and economic rights for Negroes. In 1963 racial protests and demonstrations took place in all parts of the United States. To meet the growing demands for Negroes. Kennedy asked congress to pass legislation requiring hotels, motels and restaurants to admit customers regardless of race. The president said: <<The time has come for the congress of the United States to [...] make clear to all that race has no place in American life or law>>. In 1963 President Kennedy was assassinated at Dallas, Texas.

Why is he better known than presidents who lived longer and did much more? Perhaps you can find the answer in this speech: <<Let every nation know [...] that we shall pay any price [...] support any friend, oppose any foe, in order to assure the survival and the success of liberty>>.

Unfortunately, it was in the name of freedom that he led his country into the cruel war in Vietnam. But he paid a terrible price for his love of liberty.

\section{Martin Luther King}
Martin Luther King was an American civil rights leader who worked to achieve social, political and economic equality for Negroes by peaceful means.

He was born in Atlanta, Georgia, in 1992. His father was a Baptist minister and he himself was ordained a minister in 1947. He had grown up in the South, the center of racial segregation. There the Negroes were excluded from the places frequented by Whites, they had to occupy different seats in buses, trains and had to go to separate schools/

After seeing so much injustice, Martin Luther King could have joined those who wanted a violent revolution. But he decided to lead the Negroes struggle for equality through "non violent resistance". He began his crusade in 1955 leading a boycott if buses in Montgomery and then he led demonstrations in many parts of the country. His non-violent program reached a high point in 1963  when 200000 whites and black people marched peacefully in Washington. In 1964 he was awarded the Nobel Peace Prize and as a result of his peaceful struggle, congress enacted the Civil Rights Act.

Only four years later, on April $4$th 1968, he was shot dead by a white gunman. The following words from a spiritual are written on his tombstone in Atlanta: <<Free at last, free at last, thank God. I'm free at last>>.

From one of his speeches: <<We will return good for evil. Christ showed us the way and Gandhi showed us it could work>>.
