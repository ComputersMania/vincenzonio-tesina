Genere musicale nato negli Stati Uniti alla fine dell'Ottocento e sviluppatosi in tutto il mondo, in origine, fu un’arte tipica ed esclusiva delle comunità dei neri di origine africana. Dato che già in origine il jazz si frammentò in numerosi stili diversi, più che una singola definizione è possibile avanzare qualche generalizzazione.

Elemento base del jazz è l’improvvisazione. Ciò significa che i musicisti jazz non seguono una partitura, come fanno invece gli esecutori di brani di musica definita “classica”, bensì sviluppano liberamente la melodia sulla base di una progressione di accordi e di un insieme di regole ritmiche e armoniche.

Il jazzista improvvisa all'interno delle convenzioni dello stile adottato. Di norma, l'improvvisazione segue il giro armonico di una canzone preesistente o di una composizione originale. I musicisti imitano gli stili vocali dei cantanti, con l'uso del glissando, sfumature di altezza leggiadre (come le cosiddette "blue notes", le note leggermente bemollizzate nella scala del blues), e altri effetti.

Il ritmo è caratterizzato dall'uso costante del sincopato (con accenti in posizioni impreviste) e dallo "swing": una sensazione di spinta trascinante dovuta al fatto che la melodia viene percepita ora insieme, ora leggermente sfasata rispetto all'attesa scansione della misura. Le partiture scritte, quando ci sono, fungono meramente da guida, fornendo la struttura in cui inserire l'improvvisazione. La strumentazione tipica ha come nucleo una sezione ritmica costituita da pianoforte, contrabbasso, batteria e a volte chitarra, alla quale si può aggiungere la più grande varietà di strumenti. Nelle grandi orchestre, i fiati sono raggruppati in tre sezioni: sassofoni, tromboni e trombe. Anche la chitarra, sia elettrica sia acustica, ha avuto un ruolo importante. Con il passare del tempo altri strumenti hanno fatto la loro comparsa nel jazz, ad esempio il violino e persino alcuni strumenti elettronici.

Il jazz si basa sul principio che alla progressione di accordi di qualsiasi canzone si può adattare un numero infinito di melodie. Il musicista improvvisa nuove melodie che rispondono a quel giro armonico, il quale viene ripetuto a ogni intervento di un nuovo solista. Il jazz è un genere musicale che lascia grande libertà agli interpreti, ma che al tempo stesso richiede al jazzista una profonda conoscenza della musica e doti di grande originalità sul piano dell’esecuzione. I modelli formali più frequenti sono quelli del song e del blues.

Un altro elemento fondamentale del jazz è la voce, tanto che tra le grandi figure che hanno fatto la storia di questo genere musicale troviamo moltissimi cantanti. Basti pensare a Ella Fitzgerald, e a Louis Armstrong, quest’ultimo grande trombettista e al tempo stesso cantante straordinario.

Armstrong, infatti, era uno straordinario improvvisatore e contribuì a diffondere in tutto il mondo il jazz. Modificò la scena del jazz portando il solista in primo piano, e nei suoi gruppi di incisione, gli Hot Five e gli Hot Seven, dimostrò che l'improvvisazione nel jazz poteva andare ben al di là dei semplici abbellimenti e arrivare a creare nuove melodie a partire dalla successione armonica del motivo iniziale. Fissò anche il modello per tutti i cantanti successivi, non solo per il modo in cui alterava le parole e le melodie delle canzoni, ma anche improvvisando senza parole, usando la voce come uno strumento, con la tecnica dello scat.
