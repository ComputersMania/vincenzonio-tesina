Il Jazz nacque a New Orleans da una commistione fra blues e sonorità di influenza europea ed africana, e conobbe la sua maggiore popolarità soprattutto a partire dagli anni '20.

Durante gli anni' 30 dal jazz si sviluppò un nuovo genere musicale, lo swing, caratterizzato da una spensieratezza e da uno ritmo "dondolante" (lo swing, appunto). Questo genere musicale divenne molto popolare durante gli anni della Seconda Guerra Mondiale e questo fu l'unico momento storico in cui il jazz, tradizionalmente considerato come musica "colta", riuscì a raggiungere le masse, conoscendo una diffusione che mai si era avuta e che mai più si è ripetuta.

La musica jazz e la Seconda Guerra Mondiale ebbero una forte e reciproca influenza: la rivista americana "Down Beat" pubblicò queste parole "I musicisti oggi, non sono soltanto suonatori di jazz, loro sono i soldati della musica", questo perché il jazz contribuì a tenere alto il morale dei soldati impegnati sul fronte di guerra e, nello stesso tempo, fu di conforto per i loro cari rimasti ad attenderli in patria. La musica e i pochi divertimenti che i soldati si concedevano al fronte erano forieri di cari ricordi che aiutavano e motivavano i giovani ad impegnarsi per riuscire a tornare a casa il più presto possibile. Inoltre la musica era certamente utile per distrarsi e trovare un momento di distacco dagli orrori che venivano vissuti quotidianamente sui campi di battaglia.

In quegli anni molti musicisti jazz si arruolarono volontariamente nell'esercito, altri, invece, portarono i loro concerti negli Stati Uniti o, quando possibile, anche all'estero, contribuendo a diffondere i valori americani nel mondo. Se però il jazz ebbe un impatto importante sulla guerra, anche la guerra influì sul genere musicale allora più popolare in America.

Intanto, a causa dei razionamenti della benzina, viaggiare era diventato più difficoltoso: c'erano meno autobus a disposizioni delle orchestre e i treni, spesso, erano occupati dai soldati che partivano o tornavano dall'Europa. In quegli anni, inoltre, una forte tassa sugli intrattenimenti serali decretò la chiusura di molte sale da ballo americane che non riuscivano più a sostenere i costi di gestione, in un momento economicamente già piuttosto difficile. In più, nel 1942, la Federazione Americana dei Musicisti impose un divieto di registrazione agli artisti(il Recording Ban), almeno fino a quando le case discografiche non si fossero impegnate nel pagamento dei diritti d'autore per le canzoni suonate alla radio e nei juke-box.

Fu in questo periodo abbastanza difficile della musica jazz, che si impose un nuovo genere: il bebop. Dizzie Gillespie e Charlie Parker ne furono gli iniziatori, ben presto seguiti da molti altri musicisti. Caratteristica del bebop sono i tempi molto veloci, le dissonanze armoniche e l'improvvisazione, vera anima di questo genere musicale astratto e ribelle.
