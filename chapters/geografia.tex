\section{Il territorio e il clima}
Cuba è la più vasta isola dei Caraibi; appartiene all'arcipelago delle Grandi Antille. E' costituita in gran parte da un'ampia {\bf pianura} calcarea. Le {\bf zone montuose} sono intervallate da ampie e fertili vallate. La {\bf costa} è generalmente bassa e paludosa. Al territorio cubano appartengono inoltre più di 1600 isole minori.

Il {\bf clima} è caldo e umido, ma mitigato dall'influenza del mare. Tra agosto e ottobre l'arcipelago è soggetto a cicloni provenienti da sud-ovest.

\section{Popolazione e città}
Oltre un terzo della popolazione è di origine bianca, per le consistenti migrazioni dall'europa, sono presenti mulatti, neri e asiatici.

Oltre il 75\% dei cubani vive nelle città. L'Avana, la capitale, è la principale città dell'america centrale e importante centro di riferimento culturale di tutta l'America latina. Conserva il fascino dell'antico centro coloniale spagnolo, con palazzi e chiese; è inoltre un grande porto commerciale.

\section{Economia}
L'economia è sotto il controllo dello stato, goernato da un regime comunista; negli ultimi anni, tuttavia, si sono introdotte riforme che lasciano maggiori spazi all'impresa privata e alle piccole aziende.

Settore tradizionalmente trainante dell'economia cubana, l'agricoltura conserva tuttora grande importanza. La canna da zucchero e il tabacco sono le coltivazioni principali. Discreti sono l'allevamento bovino e la pesca.

Le risorse del sottosuolo (petrolio, cobalto e rame) sono modeste.

L'industria comprende fabbriche per la trasformazione dei prodotti agricoli per la raffinazione del petrolio, la cui produzione resta però insufficente.

Nel terziario è in espansione il turismo.
