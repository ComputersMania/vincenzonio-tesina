Vincere, vincere, vincere. Che sia in americano o in russo, questo è stato l’imperativo degli atleti dei due blocchi durante le Olimpiadi del dopoguerra.

Helsinki 1952, quindicesima Olimpiade

A Helsinki Stalin decide di far partecipare anche la sua nazione, ma la Finlandia era la stessa terra che aveva combattuto contro l’Armata Rossa e inoltre era oggetto da tempo delle mire espansionistiche russe: con l’avvicinarsi dei Giochi quindi le tensioni politiche si moltiplicano sempre di più.

L’attrito si manifesta quando viene chiesto ai sovietici di far passare la fiaccola attraverso i Paesi Baltici, permesso che la Russia nega, costringendo l’organizzazione finlandese a cambiare il percorso attraverso i Paesi Scandinavi.

Un altro momento cruciale avviene pochi giorni prima dell’inizio della manifestazione quando le Autorità di Mosca avanzano la pretesa di porre come base per i loro atleti Leningrado e non la capitale finlandese. Alla fine le due Nazioni trovano un compromesso: gli atleti russi avrebbero avuto un villaggio olimpico separato dagli altri.

Le Olimpiadi di Helsinki segnano non solo l’esordio della Russia ma anche la riammissione della Germania e del Giappone.

La congregazione sovietica, alla sua prima partecipazione, riesce a portare in patria 71 medaglie contro le 76 degli USA.
