\section{La vita}
Giuseppe Ungaretti nacque ad Alessandria d’Egitto l’8 febbraio 1888 da genitori lucchesi emigrati in cerca di lavoro; studiò in una scuola di lingua francese della città egiziana.

Nel 1912 si trasferì a Parigi, dove frequentò l’Università della Sorbona e incontrò alcuni tra gli esponenti più importanti della cultura europea del tempo. Qui approfondì la conoscenza dei poeti simbolisti come Charles Baudelaire e Stephane Mallarmé, che esercitarono su di lui un’influenza fondamentale.

Nel 1914 si trasferì in Italia, dove, arruolatosi volontario come soldato semplice di fanteria, partecipò alla Prima guerra mondiale combattendo sul fronte del Carso.  Dall’esperienza diretta delle atrocità della guerra, drammaticamente vissute in prima persona, prese forma il primo nucleo della sua produzione poetica. Nacquero così le raccolte Il porto sepolto (1916) e Allegria di naufragi (1919); le poesie furono poi riunite nel volume L’Allegria (1931).

Al termine del conflitto, Giuseppe Ungaretti visse a Parigi per un anno, come corrispondente del giornale fondato da Benito Mussolini, "Il popolo d’Italia". L‘adesione al fascismo nasceva dall’ingenua fiducia nel rinnovamento economico e spirituale del popolo italiano che il regime prometteva attraverso la massiccia propaganda.

Tra il 1920 e il 1936 Giuseppe Ungaretti svolse un’intensa attività di giornalista e conferenziere, viaggiando molto in Italia e in Europa. Nel 1933 uscì la raccolta Sentimento del Tempo.

Nel 1936 Giuseppe Ungaretti accettò la cattedra di Lingua e letteratura italiana presso l’Università di San Paolo, in Brasile, dove andò a vivere con la moglie e i due figli. Qui lo colpirono due gravi lutti familiari: la morte del fratello Costantino e quella del figlio Antonietto, prematuramente scomparso all’età di soli dieci anni.

Il ritorno in Italia nel 1942 coincise con la Seconda guerra mondiale. Alla tragedia privata si sovrappose così quella pubblica e questo duplice dramma ispirò la raccolta emblematicamente intitolata Il Dolore (1947). Fu nominato Accademico d’Italia e ottenne "per chiara fama" la cattedra di Letteratura italiana moderna e contemporanea all’Università di Roma.

Pubblicò le raccolte La Terra Promessa (1950), Un Grido e Paesaggi (1952), Il taccuino del vecchio (1960), Dialogo (1968) e le traduzioni di alcuni importanti autori di lingua inglese, francese, spagnola.

Nel 1970 fu colto da malore durante un viaggio negli Stati Uniti e, rientrato in Italia, morì a Milano, il 2 giugno, per broncopolmonite, all’età di ottantadue anni.

\section{Soldati}

\begin{verse}
  Si sta come \\
  d'autunno \\
  sugli alberi \\
  le foglie
\end{verse}

La lirica è di Giuseppe Ungaretti, tratta da Vita di un uomo.

Questa poesia è composta da solamente un periodo costituito da quattro versi senza rime e in più brevissimi, ma che comunicano con la loro essenzialità sintattica significati molto più profondi.

\noindent
Sarebbe difficile comprendere il significato della lirica senza leggere il titolo “Soldati”.

Troviamo già all’inizio una similitudine e questo è il primo termine di paragone. La lirica esprime quel filo invisibile tra la vita e la morte in cui si trovano i soldati, cioè come foglie sugli alberi in autunno che cadono con un soffio di vento: la morte. Come le foglie nascono e muoiono, allo stesso modo fanno gli uomini.

E’ significativo l’enjambement dopo il come che rende ancora meglio l’idea di stabilità comunicata dal verbo stare.

In questa lirica, come in parecchie altre riferite alla guerra (Risvegli, Veglia, San Martino del Carso) Ungaretti esprime lo stato d’animo in cui ci si trova spesso in guerra, in questo caso la tensione della morte imminente.

Questo breve componimento di Giuseppe Ungaretti  si trova nella raccolta L’ Allegria, più specificatamente nella parte dell’ opera intitolata Girovago. Questa poesia è formata un’unica similitudine, soldati/foglie; dal punto di vista metrico, la lirica presenta due settenari divisi in quattro versi e un enjambement tra il primo e il secondo verso.

Leggendo il testo notiamo subito come quest’ultimo, insieme a moltissimi altri presenti nella medesima raccolta, si riferisca alla guerra, e sia attraversato da un presagio di morte.

Perché, dunque, chiamare L’allegria una raccolta di poesie in cui prevalgono tali temi?

Ungaretti spiega come il sentimento dell’allegria, in questo caso, scaturisca nell’attimo in cui l’uomo realizza di essere scampato alla morte. L’esperienza diretta che il poeta fa della guerra durante il primo conflitto mondiale, la quotidiana tensione verso la vita nell’atto pratico della sopravvivenza, porta al culmine tale sentimento. Soldati rientra certamente in questo filone tematico. Composto nel 1918 mentre Ungaretti si trovava in trincea nel bosco di Courton, esprime il dramma e la precarietà del momento storico e della condizione umana. I soldati vengono qui paragonati a foglie autunnali che, ancora appese agli alberi, procedono inevitabilmente verso la caduta e la morte, vittime dello scorrere del tempo.

Al termine “soldati” è però facilmente sostituibile quello di uomini, e alla guerra è applicabile la più ampia nozione di vita. Così ci rendiamo conto come non siano solo i militari al fronte a vivere una condizione precaria e incerta, ma come sia la natura stessa dell’essere umano a dover fare i conti con la propria finitudine.

Il parallelismo tra uomo e foglie, immagine molto riuscita, non è una scelta letteraria innovativa operata da Ungaretti, ma possiamo ritrovarla in testi poetici anche molto antichi, ad esempio nell’Iliade.
