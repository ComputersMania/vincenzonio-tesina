\section{I due blocchi}
Dopo la fine della seconda guerra mondiale nel mondo si affermarono due grandi superpotenze: gli Stati Uniti e l'Unione Sovietica.

Gli Stati Uniti avevano incrementato la capacità produttiva della loro industria, per rifornire gli eserciti propri e degli alleati; nelle casse dello stato si trovavano i due terzi dell'oro mondiale e il dollaro era considerato la moneta base degli scambi mondiali. Dal punto di vista tecnologico e della ricerca scientifica erano all'avanguardia, avendo fatto uso della bomba atomica arma che nessun altro esercito possedeva.

L'URSS vantava il più grande esercito di terra del mondo, l'Armata Rossa. Inoltre in tutti i paesi europei c'erano partiti comunisti che guardavano all'URSS di Stalin come a un modello.

I sistemi economici e politici delle due superpotenze erano opposti tra loro in quanto gli stati uniti erano una democrazia liberale e la loro economia era capitalistica, mentre l'Unione Sovietica era uno stato totalitario sotto il potere del partito comunista e la sua economia era pianificata dallo stato. Nei territori liberati dall'armata rossa Stalin impose regimi detti "democrazie popolari" in cui il potere fu preso dal partito comunista e ai quali fu imposta l'interruzione di ogni rapporto con i paesi non comunisti dell'occidente. Invece i governi Americani istituirono un grandioso piano di aiuti economici per i Paesi europei danneggiati dalla guerra detto piano Marshall.

\section{La nato e il patto di Varsavia}
Nel 1949 gli USA costituirono la NATO, un'alleanza militare di difesa; in contrapposizione l'URSS stabilì nel 1955 il patto di Varsavia, con i paesi europei a regime comunista.

Si formarono così due blocchi opposti di paesi alleati:
\begin{itemize}
  \item Il blocco occidentale comprendeva Portogallo, Gran Bretagna, Francia, Paesi Bassi, Belgio, Lussemburgo, Italia, Norvegia, Grecia, Turchia, USA e Canada;
  \item Il blocco comunista che comprendeva Estonia, Lettonia, Lituania, Bielorussia, Finlandia, Polonia, Cecoslovacchia, Romania e Bulgaria.
\end{itemize}
I due blocchi seguivano rispettivamente le ideologie dell'USA e degli URSS.

I rapporti tra gli Stati Uniti e l'Unione Sovietica furono spesso molto tesi, ma i due stati non dichiararono mai guerra; questo periodo che durò qurant'anni fu definito guerra fredda.

\section{La crisi missilistica cubana}
Durante il 1962 Kennedy, il presidente America e Chruščëv furono Astolfo

\section{Il muro di Berlino}
