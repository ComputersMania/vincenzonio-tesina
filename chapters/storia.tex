\section{I due blocchi}
Dopo la fine della seconda guerra mondiale nel mondo si affermarono due grandi superpotenze: gli Stati Uniti e l'Unione Sovietica.

Gli Stati Uniti avevano incrementato la capacità produttiva della loro industria, per rifornire gli eserciti propri e degli alleati; nelle casse dello stato si trovavano i due terzi dell'oro mondiale e il dollaro era considerato la moneta base degli scambi mondiali. Dal punto di vista tecnologico e della ricerca scientifica erano all'avanguardia, avendo fatto uso della bomba atomica arma che nessun altro esercito possedeva.

L'URSS vantava il più grande esercito di terra del mondo, l'Armata Rossa. Inoltre in tutti i paesi europei c'erano partiti comunisti che guardavano all'URSS di Stalin come a un modello.

I sistemi economici e politici delle due superpotenze erano opposti tra loro in quanto gli stati uniti erano una democrazia liberale e la loro economia era capitalistica, mentre l'Unione Sovietica era uno stato totalitario sotto il potere del partito comunista e la sua economia era pianificata dallo stato. Nei territori liberati dall'armata rossa Stalin impose regimi detti "democrazie popolari" in cui il potere fu preso dal partito comunista e ai quali fu imposta l'interruzione di ogni rapporto con i paesi non comunisti dell'occidente. Invece i governi Americani istituirono un grandioso piano di aiuti economici per i Paesi europei danneggiati dalla guerra detto piano Marshall.

\section{La nato e il patto di Varsavia}
Nel 1949 gli USA costituirono la NATO, un'alleanza militare di difesa; in contrapposizione l'URSS stabilì nel 1955 il patto di Varsavia, con i paesi europei a regime comunista.

Si formarono così due blocchi opposti di paesi alleati:
\begin{itemize}
  \item Il blocco occidentale comprendeva Portogallo, Gran Bretagna, Francia, Paesi Bassi, Belgio, Lussemburgo, Italia, Norvegia, Grecia, Turchia, USA e Canada;
  \item Il blocco comunista che comprendeva Estonia, Lettonia, Lituania, Bielorussia, Finlandia, Polonia, Cecoslovacchia, Romania e Bulgaria.
\end{itemize}
I due blocchi seguivano rispettivamente le ideologie dell'USA e degli URSS.

I rapporti tra gli Stati Uniti e l'Unione Sovietica furono spesso molto tesi, ma i due stati non dichiararono mai guerra; questo periodo che durò qurant'anni fu definito guerra fredda.

Durante gli anni della guerra fredda le due superpotenze entrarono in competizione per rinnovare e ingrandire i propri arsenali, scatenando una vera e propria corsa agli armamenti.

\noindent
La guerra fredda è stato un processo molto complesso ed alcuni dei suoi avvenimenti sono i più famosi.

\section{La guerra di Corea}
Il primo scontro militare avvenne in Corea, paese diviso nella parte settentrionale governata dai comunisti sostenuti dall’ URSS e la parte meridionale sotto un governo nazionalista appoggiato dagli USA. I due governi coreani non accettarono la divisione e nel 1950 l’esercito nord-coreano invase il sud. Gli USA intervennero spingendosi fino al territorio del nord.

\noindent
Solo nel 1953 viene firmato un armistizio che rendeva definitiva la divisione fra le due Coree.

\section{La crisi missilistica cubana}
Durante il 1962 Kennedy , il presidente americano, e Chruscev, il successore di Stalin, furono protagonisti della crisi di Cuba. Nel 1959 il rivoluzionario Fidel Castro aveva rovesciato una dittatura militare americana preoccupando gli Stati Uniti. Per indebolire il governo di Castro gli USA appoggiarono un tentativo di sbarco di controrivoluzionari cubani , ma la spedizione fallì e Castro proclamò Cuba una Rebubblica socialista appoggiata dall’ URSS.

Nel 1962 l’ URSS installò nell’isola alcune basi militari per missili atomici, provocando da parte degli Stati Uniti un blocco navale intorno a Cuba. Dopo qualche giorno però i sovietici smantellarono le basi e gli Stati Uniti si impegnarono a rispettare l’ indipendenza di Cuba.

\section{Il muro di Berlino}
La Germania era occupata dalle nazioni vincitrici della seconda guerra mondiale. In particolar modo Berlino era divisa in due parti. Berlino ovest ,controllata da Gran Bretagna ,Francia e Stati Uniti, era una città prospera, libera e culturalmente all’avanguardia, il contrario di Berlino est che era sotto l’influsso sovietico. Così, per porre fine alle migrazioni dalle parte est alla parte ovest del paese, nel 1961 il governo comunista della Berlino orientale fece costruire un muro alto quattro metri e lungo 47 chilometri. Il muro di Berlino divenne il simbolo più significativo della guerra fredda, simboleggiando il carattere oppressivo e antidemocratico del regime comunista e mettendo in risalto il contrasto con la politica adottata dall'occidente.
